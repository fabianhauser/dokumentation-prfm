% !TeX spellcheck = en_GB
% /*
%  * ----------------------------------------------------------------------------
%  * "THE BEER-WARE LICENSE" (Revision 42):
%  * <michi.wieland@hotmail.com> wrote this file. As long as you retain this notice you
%  * can do whatever you want with this stuff. If we meet some day, and you think
%  * this stuff is worth it, you can buy me a beer in return. Michael Wieland
%  * ----------------------------------------------------------------------------
%  */

\input{./template.tex}

% Dokumentinformationen
\newcommand{\SUBJECT}{Zusammenfassung}
\newcommand{\TITLE}{Programmieren und Formale Methoden}

\loadglsentries{glossar}

\input{./front.tex}

\section{Introduction}
The use of a formal language is to describe and solve problems is central to computer science.

\begin{description}
	\item[Formal Methods] Make computer programming more reliable and predictable by mathematical analysis.
	\item[Formal Language] Set of string of symbols that are constrained by specific rules. Membership in this set can automatically be determined (mechanical processing possible)
	\item[Informal Language] Any living natural language
\end{description}

\section{Programming Paradigms}

Paradigm is a Worldview / Concept / Tought patterns.

OOP uses multiple of the paradigms below, but is not a paradigm in this sense.

\subsection{The Imperative Programming Paradigm}

Is imperative (de: Befehlend) and focuses on \emph{how} a program works, like a recipee. It uses commands to change the program's state.

A program (in this course) is a sequence of commands and the computation is a change of state.

Most hardware implementations of computer is imperative.

It is based on the Von Neumann architecture.

\subsubsection{Basic Building Blocks}

\begin{description}
	\item[Assignments (sate/store/heap)] x:= x+1 %TODO: COde
	\item[Sequential composition] ...; ...
	\item[Conditional excetion] if ... then ... else...
	\item[Repetition] while .. do .. / goto ...
\end{description}



\subsection{The Declarative Programming Paradigm}

Discribe logic of computation without expicitly describing its control flow.

\emph{What} should a programm accomplish, not how the language should do smth. (But it must still be computable in the end).

Programs are here theories of a formal logic. Computations are deductions.

Examples: Spreadsheets, Regex, Query Languages (SQL etc.), Functional programming languages (Haskel, ...), Logic programming (Prolog)

\subsubsection{The Functional Programming Paradigm}

Basis: The Lambda Calculus

Programs are function definitions, computations are function evaluations.

\subsubsection{The Logical Programming Paradigm}
Basis: Predicate logic

Programs are rules in predicate logic and computation is a proof of a predicate.



\section{Formal Proof (Propositional and Predicate logic)}

\begin{figure}
	\centering
	\includegraphics[width=0.7\linewidth]{images/sequent_calculus_overview}
	\caption{Overview calculus proof family.}
	\label{fig:sequentcalculusoverview}
\end{figure}

\subsection{refresher ''Diskrete Mathematik''}
Kinds of logic: Boolean, ''Prädikatenlogik'' ($\forall,\exists, \Rightarrow, \Leftrightarrow, \wedge, \vee, \neg$)

Proofs: induction, direct, indirect, by contradiction

\subsection{What are formal and informal proofs?}
A \emph{informal proof} are arguments presented as explanations in natural language (like most mathematical textbooks; need a human to check)

A \emph{formal proof} is constructed using well-formed formulas of a formal language and formally defined rules of inference.
 They can be checked by a machine and sometimes be mechanically constructed. Computer programs are able to assist proving (e.g. like Excel), sometimes a proof is even not practically possible without them.

\subsection{Sequent Calculus style}

Most programming languages are defined in the sequence calculus style; often used for types and programming languages, semantics and proofs of languages.


\subsubsection{Proof Rules}

 A device to construct proofs.
 
 \[
 \frac{\overrightarrow{A}}{C} \boldmath{r}
 \]
 
 \begin{description}
	 	\item[$r$] Rule name
	 	\item[$C$] Consequent (to be proven)
	 	\item[$\vec{A}$] list of sequent proof rules (vector).
 	\end{description}
 	
 
 We want to proof $C$ out of multiple sequent proof rules $A_1, A_2, ..., A_n$ with the rule $r$. If there is no $A$ above the line, the proof rule is a axiom. The order of proof rules $A$ is important and not permutable.
 
 To build a complete proof, all proof rules must be proven up to a axiom.


\subsubsection{Definitions}

\begin{description}
	\item[Sequent] a proof rule or axiom
	\item[Proof Rule] A rule $r$ to proof a consequent $C$ with more proof rules $A$
	\item[pending sub-goal] A sequent which is not yet proven by a axiom.
	\item[Axiom] Proof rule without sequent $A$; assumed to be true. \\
	There should be as less as possible axioms.
	\item[Calculus (or Theory)] (possibly infinite) set of proof rules $r_1$, $r_2$.
	\item[Proof (or Derivations)] Ordered proof rules in a tree, so that all proof rules are proven by a axiom.
\end{description}

\subsubsection{Example}

\paragraph{Type checking proof}

of the shorthand if else construct for integers.

\[
	\frac{
			\frac{
				\frac{}{x::int}x_{int} \text{   } \frac{}{y::int}y_{int}
			}{(x > y)::bool} \text{   }
		  \frac{
				\frac{}{x::int}x_{int} \text{   } \frac{}{y::int}y_{int}
			}{(x - y)::int} -_{int1} \text{   }
			\frac{
				\frac{}{y::int}y_{int} \text{   } \frac{}{x::int}x_{int}
			}{(y - x)::int} -_{int2}
		}{(x>y? x - y : y - x)::int} condExpr_{int}
\]

\subsection{Propositional Calculus}


\section{Logic programming (Prolog)}

\section{Lambda calculus}

\section{Functional Programming (Haskell)}

\section{Multi-Paradigm Programming (Scala)}


\input{./appendix.tex}
