% !TeX spellcheck = en_GB
% /*
%  * ----------------------------------------------------------------------------
%  * "THE BEER-WARE LICENSE" (Revision 42):
%  * <michi.wieland@hotmail.com> wrote this file. As long as you retain this notice you
%  * can do whatever you want with this stuff. If we meet some day, and you think
%  * this stuff is worth it, you can buy me a beer in return. Michael Wieland
%  * ----------------------------------------------------------------------------
%  */

\input{./template.tex}

% Dokumentinformationen
\newcommand{\SUBJECT}{Zusammenfassung}
\newcommand{\TITLE}{Programmieren und Formale Methoden}

\loadglsentries{glossar}

\input{./front.tex}

\section{Introduction}
The use of a formal language is to describe and solve problems is central to computer science.

\begin{description}
	\item[Formal Methods] Make computer programming more reliable and predictable by mathematical analysis.
	\item[Formal Language] Set of string of symbols that are constrained by specific rules. Membership in this set can automatically be determined (mechanical processing possible)
	\item[Informal Language] Any living natural language
\end{description}

\section{Programming Paradigms}

Paradigm is a Worldview / Concept / Tought patterns.

OOP uses multiple of the paradigms below, but is not a paradigm in this sense.

\subsection{The Imperative Programming Paradigm}

Is imperative (de: Befehlend) and focuses on \emph{how} a program works, like a recipee. It uses commands to change the program's state.

A program (in this course) is a sequence of commands and the computation is a change of state.

Most hardware implementations of computer is imperative.

It is based on the Von Neumann architecture.

\subsubsection{Basic Building Blocks}

\begin{description}
	\item[Assignments (sate/store/heap)] x:= x+1 %TODO: COde
	\item[Sequential composition] ...; ...
	\item[Conditional excetion] if ... then ... else...
	\item[Repetition] while .. do .. / goto ...
\end{description}



\subsection{The Declarative Programming Paradigm}

Discribe logic of computation without expicitly describing its control flow.

\emph{What} should a programm accomplish, not how the language should do smth. (But it must still be computable in the end).

Programs are here theories of a formal logic. Computations are deductions.

Examples: Spreadsheets, Regex, Query Languages (SQL etc.), Functional programming languages (Haskel, ...), Logic programming (Prolog)

\subsubsection{The Functional Programming Paradigm}

Basis: The Lambda Calculus

Programs are function definitions, computations are function evaluations.

\subsubsection{The Logical Programming Paradigm}
Basis: Predicate logic

Programs are rules in predicate logic and computation is a proof of a predicate.



\section{Formal Proof (Propositional and Predicate logic)}
%TODO: Folie 3 PrFm_01-FOrmal Proof_01-Formal proof covered?

\subsection{refresher ''Diskrete Mathematik''}
Kinds of logic: Boolean, Prädikatenlogik ($\forall,\exists$) %TODO

Proofs: induction, direct, indirect, by contradiction

\subsection{What are formal and informal proofs?}
Informal Proof. Arguments are presented as explanations in nautral language (most math textbooks; need a human to check)

Formal Proof: Is constructed using well-formed formulas of a formal language and formally defined rules of inference.
 Can be checked by a machine, proofs can sometimes be mechanically constructed. Computer programs are able to assist proof (e.g. like excel), sometimes not practically possible without them.

\subsection{Sequent Calculus style}

Most Programming Languages are defined in the sequence calculus style; often used for types and programming languages, semantics and proofs of languages.

\begin{figure}
\centering
\includegraphics[width=0.7\linewidth]{images/sequent_calculus_overview}
\caption{Overview Sequent Calculus chapter}
\label{fig:sequentcalculusoverview}
\end{figure}

\subsubsection{Definitions}

\begin{description}
	\item[Sequent] Something that we want to proof (e.g. $S1$)
	\item[Proof Rule] A device to construct proofs of sequents. Name: $r$, $C$: Consequent, $\overrightarrow{A}$: list (vector).
	$\frac{\overrightarrow{A}}{C} \boldmath{r}$ \\
	We want to Proof C out of multiple Proofs $A_1$, $A_2$ etc. with the rule $r$. If there is no A above the line, it is a axiom.
	\item[Axiom] Proof Rule without A %TODO Slide 5
	\item[Theory] (possibly infinite) Set of proof rules $r_1$, $r_2$, etc.
	\item[Proof (rules)] Must be Ordered in a Tree until we get to a Axiom. %TODO: Slide 6
\end{description}

\section{Logic programming (Prolog)}

\section{Lambda calculus}

\section{Functional Programming (Haskell)}

\section{Multi-Paradigm Programming (Scala)}


\input{./appendix.tex}
